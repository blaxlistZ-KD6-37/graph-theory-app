\providecommand{\graphTheoryPreambleLoaded}{}
\ifx\graphTheoryPreambleLoaded
\documentclass{article}
\usepackage{./1_Preamble/graph_theory_preamble}

\begin{document}
	\fi
	
	\subsection*{Objectives}
	\addcontentsline{toc}{subsection}{Objectives}
	The primary aim for this research is to see how essential graph theory is in video-games in general. More specifically, Grand Theft Auto San Andreas, we will analyze the given region using some graph theory principles, concepts, and algorithms. The following principles that we shall apply for the video-game will include:
	\begin{enumerate}
		\item A \textbf{graph model} from the chosen region from figure (\ref{ls_map}).
		\item The \textbf{order}, \textbf{size}, and \textbf{degrees} of the graph.
		\item The \textbf{traversability} of the graph.
			\begin{itemize}
				\item Is it an \textbf{Eulerian path} or \textbf{Cycle}?
				\item Is it a \textbf{Hamiltonian path} or \textbf{Cycle}?
			\end{itemize}
		\item The optimal \textbf{postman tour} for the graph.
		\item The \textbf{minimum spanning tree} of the graph.
			\begin{itemize}
				\item Using \textbf{Kruskal}'s Algorithm.
				\item Using \textbf{Prim}'s Algorithm.
			\end{itemize}
		\item The \textbf{shortest route} for the graph using \textbf{Djikstra}'s algorithm.
	\end{enumerate} At the end, we shall check any major findings that were acquired through graph model analysis, and recommend on how fans, game developers, speedrunners, knowledge-seekers shall utilize these principles in this game, other games or even real life scenarios.   
	
	\ifx\graphTheoryPreambleLoaded
\end{document}
\fi